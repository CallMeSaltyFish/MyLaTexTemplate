%-*- coding=utf-8 -*-
\documentclass[UTF8]{ctexart}
\usepackage{geometry}
\usepackage{cite}
\CTEXsetup[format={\Huge\bfseries}]{section}
\CTEXsetup[format={\LARGE\bfseries}]{subsection}
%\usepackage{titlesec}
\geometry{left=3.18cm,right=3.18cm,top=2.54cm,bottom=2.54cm}
\usepackage{graphicx}
\pagestyle{plain}	
% \usepackage{booktabs}
% \usepackage{subfigure}
\usepackage{setspace}
\begin{document}
\nocite{*}%show all references
\bibliographystyle{plain}
	\begin{center}
		\quad \\
		\quad \\
		\vskip 3.5cm
		\heiti \fontsize{45}{17} 高级网络编程
		\vskip 3.5cm
		\heiti \fontsize{39}{17} 课程设计
	\end{center}
	\vskip 3.5cm
	\begin{quotation}
		\heiti \fontsize{36}{17}
		\doublespacing
		\par\setlength\parindent{12em}
		\quad 
		\heiti		
		
		名\hspace{0.61cm} 称:\underline{基于多播的聊天室}
		
		日\hspace{0.61cm} 期:\underline{2019年6月23日}
		
		姓\hspace{0.61cm} 名:\underline{王梓岩}
		
		学\hspace{0.61cm} 号:\underline{71Y16118}
		
		\vskip 2cm
		\centering
	\end{quotation}
	
\newpage
\songti \fontsize{13}{13}
\large
\section{需求分析}
在unix环境下搭建一个聊天室,用户可在聊天室中发送文本和文件。凡是在多播组的主机,均可接收到该文本或文件。当有新用户加入或老用户退出多播组时,其余主机也可被通知到。
\section{开发环境}
Linux ubuntu 4.13.0-39-generic X64 (共计三台机器,均为该操作系统)
\section{设计思路}
多播,也叫组播,将局域网中同一业务类型主机进行了逻辑上的分组,进行数据收发的时候其数据仅仅在同一分组中进行,其他的主机没有加入此分组不能收发对应的数据。
\par
ICMP回应请求(echo-request)和应答消息(echo-reply)用于诊断两个系统(主机或路由器)之间是否多播的地址是特定的,D类地址用于多播。D类IP地址就是多播IP地址,即 224.0.0.0 至 239.255.255.255 之间的IP地址,并被划分为三类:局部多播地址:在 224.0.0.0 至 224.0.0.255 之间,这是为路由协议和其他用途保留的地址,路由器并不转发属于此范围的IP包;预留多播地址:在 224.0.1.0 至 238.255.255.255 之间,可用于全球范围(如Internet)或网络协议;
管理权限多播地址:在 239.0.0.0 至 239.255.255.255 之间,可供组织内部使用,类似于私有IP地址,不能用于Internet,可限制多播范围。本次实验我们选用 224.0.0.111 作为多播地址。
\par
首先,发送和接收必须要同时进行,而两个操作均属于卡死式的行为,因此必须要建立两个线(进)程分别负责。通讯的流程较为简单,初始化阶段需要建立用于多播的套接口,之后设置多播的参数,例如超时时间TTL,本地回环许可LOOP等。以上参数均初始化后就可以将主机加入多播组,之后由发送方发送一个消息,其他接收方将都会接收到该信息,并将其打印在屏幕上或写入文件。通信结束后主机离开多播组。
\par
考虑到当今主流的通讯软件(如QQ)文件与文本的传输是分离的,即发(收)文件的同时也可同时发(收)文本,因此文件与文本也要用不用的线程来处理,算上收发各自需要一个线程,本程序共计有4个线程并发。由于线程间通信花费较高,因此本程序直接将两者端口分离,文本由7777号端口接收,文件由7778号端口接收,省去了每次解析报文头部的麻烦。
\section{相关模块}
\subsection{主函数}
主函数的逻辑非常清晰:首先进行初始化,建立发送、接收文本和文件的四个套接口sendSock,recvSock, sendFileSock和recvFileSock;建立多播地址和本地地址;建立多播请求;获取本机在局域网的IP地址(详细方法见下);以及等待用户输入自己的用户名。
\par
之后调用两次fork()方法,将主进程分为3个进程,分别处理发文本、收文本和收文件,处理发文件的线程的建立会在后面提到,目前建立的三个线程均为卡住式线程,需要等待用户输入或从IO流中读取信息。
\subsection{接收文本}
方法recvText()用于接收文本,该模块由fork()方法分出的一个线程单独执行。必须指出的是,fork()方法会将原有的进程完全复制一份,也就是说两个线程对于同名变量不享有共同的地址。所以此时之前建立所有的套接口都各有2个,由于该进程只负责接收文本,因此可将其他无关套接口关闭,只保留用于收文本的套接口recvSock,其他进程也会做类似的处理,以后将不再阐述。
\par
接收文本的所有操作均置于一个死循环中,表示持续接收。调用recvfrom()方法,从指定端口读取IO流的信息,之后将其打印在屏幕上即可。
\subsection{发送文本}
方法sendText()用于发送文本,同样地,该方法由某一线程单独执行。由于在需求阶段指出,当有一新用户加入聊天后,组内所有成员均可知道他的到来,因此会首先调用sendto()方法发送一个内容为“用户名(IP地址) has joined.”的消息。
\par
之后进入死循环,循环内的逻辑为调用fgets()方法等待用户从键盘输入,之后通过sendto()方法将消息发送给多播组。但是有一点必须要注意,本程序为控制台程序,要发送文件就必须通过规定原语来实现,现规定文件的发送语句为:“file:// + 文件路径”,如“file:///home/aaa.txt”就表示要发送根目录下的home目录中的aaa.txt文件。因此,对于用户的每一个输入,必须先调用strcmp()方法判定前缀是否为“file://”,如果不是,说明这是普通的聊天信息,将其发送;如果是,则会触发fork()方法分出专门负责发文件的子进程,此时父进程直接等待下一次用户输入,子进程获得文件路径,即输入的字符串除去前缀的部分,之后调用sendFile()方法发送文件。
\subsection{发送文件}
方法sendFile()用于发送文件,该方法接受一个字符串类型的参数,为文件路径。规定文件的发送方式为第一个报文的前8位是文件大小(long类型),之后是文件名。后面的报文为文件内容。
\par
首先利用结构体stat获取文件大小,此时也会判定文件路径是否合法,不合法直接退出。unix操作系统下推荐使用stat来获取文件大小,此方法无需将文件装载进内存,因此效率较高。由于规定了文件大小占8个字节,因此直接将这个数字的16进制写入报文。文件名称可调用strrchr()方法,该方法可以返回文件路径字符串中最后一个“/”位置的指针,该指针以后的部分即为文件名。
\par
填写好第一个报文后,调用fopen()方法读取文件,之后按二进制方式读取文件,缓冲区的大小为2048,即每次最多读取2KB大小。调用fread()方法将读出的内容写入报文并发送。文件读取完毕后,关闭文件流和套接口,调用exit()方法结束该进程。可以发现发送文件的线程是按需请求的,这是因为考虑到发文件的频率不如发文本,没有必要将该线程一直在后台挂起。
\subsection{接收文件}
方法recvFile()用于接收文件,该方法由某一线程单独执行。与接收文本的逻辑类似,所有操作均位于死循环内。当收到报文时,首先会解析出文件大小和文件名,调用fopen()方法新建文件,为了避免重名,新建的文件一律以“接收时间 + 原文件名”的方式命名。之后收到的报文是文件内容,调用fwrite()方法将其写入文件,接收完毕后关闭该文件,等待接收下一个文件。
\subsection{退出聊天}
前面需求指出,我们希望当有人退出聊天时,其他组内成员知道他已退出,该功能可以整合进正常关闭程序的流程中。由于所有的进程都位于死循环中,因此必须采用其他方法中断这些循环。这里没有采用规定退出程序的原语,而是直接借助传统的unix退出程序的快捷键:Ctrl + C,unix为我们提供了捕获这一信号的系统调用:方法signal()。该方法有2个参数,第一个参数是信号种类,这里我们填SIGINT,第二个参数是触发的函数指针,这个函数用于进行后续操作。在本程序中,方法exitChat()作为处理退出聊天的模块。该方法主要进行两个操作:发送一个自己离开的消息给多播组,关闭所有活跃的套接口,最后调用exit()方法正常结束程序。
\subsection{获取本机IP}
考虑到虽然用户可以在进入聊天室前输入自己的昵称,但仍不能排除有重名的情况,因此会对每一条信息同时标注用户昵称和IP地址。unix操作系统下获取本机IP的方法为借助ioctl函数获取主机的全部网络接口信息。ioctl函数可以获取所在主机的全部网络接口信息,包括接口地址、是否支持广播等。遍历所有网卡信息,排除名称为lo的本地回环网卡后,就得到了当前网卡的全部信息,记录下对应的IP地址即可。
\section{实验结果分析及结论}
总计要使用三台主机,IP分别为 192.168.188.128 ,192.168.188.129和192.168.188.130 。以下用三台机器的用户名aaa、bbb和ccc代指,三台主机的控制台背景色分别为紫、白、黑,以便于区分。
但是,现在的视频文件往往较大,以下面这部动漫视频为例,该视频长达20分钟,大小“高达”100MB,此时虽然表面上传输成功,但不同程度地出现了错误。bbb主机虽然可以播放画面,但是播放期间声音会突然消失或者发出噪音,aaa主机则只能播放前几分钟的画面,之后会突然卡住,屏幕变白(图8)。
\par
究其原因,可以发现的是,两个接收到的文件均比原文件小(原大小118MB,下载所得的一个116MB,一个114MB),所以显然是发生了丢包。本次实验中,每次发送一个报文后调用usleep()方法使线程睡眠25毫秒,但这个延迟时间还是太短了,由于发送速率过快,发送方的缓冲区还没来得及发送出去就被覆写。UDP是无状态连接,没有专门的管线进行传输控制,无法检测丢包。
\par
为了保证传输完整性,可做出如下调整:一是增大报文间发送的间隔时间,但是这个方法治标不治本,可以预见的是,文件大小越大,所需的间隔时间可能就会越长,当然极限是IO的读取速率,不过届时这个时间间隔已经过大以至于每次下载文件所需的时间过长。
\par
另一种方法是将文件切片,将每个文件切成较小的部分,对每个部分分别进行传输,最后在接收端再合成一个文件。事实上现在一些著名的视频直播网站如twitch、YouTube等就是采取的这种方法,直播时每个文件只包含几个帧,这样即使是个别分片出现了丢包情况也没有关系,直接将后面的一个分片接上或适当添加一些马赛克(经常看直播的朋友应该会体会到),整体观看效果还是可以保证的。
\par
a总之,UDP作为一种不可靠连接,用于传输文本效率还是非常可观的,而且通过本实验也发现,多播的灵活性非常高。当传输的报文本身不大时,多播模式的威力还是很大的,当所有人处于同一局域网,十分推荐采用多播模式进行实时通讯。但是,当可靠性不可忽略的场合,如文件传输,此时UDP的丢包将不可忽略,虽然可采取切片的方式进行改善,但是仍不推荐用UDP传输必须确保完整性的文件(如rar、apk等)。保证传输可靠性或传输大文件,还是烦请您采用面向连接的TCP协议。

\bibliography{report}
\end{document}