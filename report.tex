%-*- coding=utf-8 -*-
\documentclass[UTF8]{ctexart}
\usepackage{geometry}
\usepackage{cite}
\CTEXsetup[format={\Huge\bfseries}]{section}
\CTEXsetup[format={\LARGE\bfseries}]{subsection}
%\usepackage{titlesec}
\geometry{left=3.18cm,right=3.18cm,top=2.54cm,bottom=2.54cm}
\usepackage{graphicx}
\pagestyle{plain}	
% \usepackage{booktabs}
% \usepackage{subfigure}
\usepackage{setspace}
\begin{document}
\nocite{*}%show all references
\bibliographystyle{plain}
	\begin{center}
		\quad \\
		\quad \\
		\vskip 3.5cm
		\heiti \fontsize{45}{17} 实习报告
		\vskip 3.5cm
		%\heiti \fontsize{39}{17} 课程设计
	\end{center}
	\vskip 3.5cm
	\begin{quotation}
		\heiti \fontsize{36}{17}
		\doublespacing
		\par\setlength\parindent{12em}
		\quad 
		\heiti		
		
		姓\hspace{0.61cm} 名:\underline{王梓岩}
		
		学\hspace{0.61cm} 号:\underline{71Y16118}

		学\hspace{0.61cm} 院:\underline{软件学院}
		
		实习单位:\underline{National Instruments}
		
		\vskip 2cm
		\centering
	\end{quotation}
	
\newpage
\songti \fontsize{13}{13}
\large
\section{实习的性质与意义}
本科生参与实习,可以透过理论联系实际,巩固所学的知识,提高处理实际问题的潜力,为顺利毕业进行做好充分的准备,并为自身能顺利与社会环境接轨做
准备。透过实习,进一步理解和领会所学的基本理论,了解计算机技术和信息管理技术的发展及应用,较为系统地掌握计算机应用技能和信息管理技能,把所
学知识与解决实际问题相联系,能够利用计算机处理工作中的各种信息,培养发现问题、分析问题和解决问题的潜力,从而提高从事实际工作的潜力。
\par
生产实习是一个极为重要的实践性教学环节。透过实习,使学生在社会实践中接触与本专业相关的实际工作,增强感性认识,培养和锻炼学生综合运用所学的
基础理论、基本技能和专业知识,去独立分析和解决实际问题的潜力,把理论和实践结合起来,提高实践动手潜力,为学生毕业后走上工作岗位打下必须的基
础;同时能够检验教学效果,为进一步提高教育教学质量,培养合格人才积累经验。计算机是一门对实践要求较高的学科,透过专业实习,使学生能熟悉有关
计算机专业的各个领域,使学生毕业后能胜任与本专业相关的工作。
\par
大学本科期间,主要学习的是理论基础知识,辅以实践环节,由于缺乏对当今社会市场的了解,从而无法有效地将所学知识运用到实际的生产生活中去。此时,
毕业实习的好处就凸显出来,通过亲身参与毕业实习,学生有机会接触第一线的计算机行业的工作内容及流程,有条件进行进一步的深造,从而给高校学生一
个了解社会,增加经验,熟悉工作单位的机会。锻炼自身的动手潜力,将学习的理论知识运用于实践当中,反过来还能检验书本上理论的正确性,有利于融会
贯通。同时,也能开拓视野,完善自身的知识结构,到达锻炼潜力的目的。一切都是为了让实践者对本专业知识构成一个客观,理性的认识,从而不与社会现
实相脱节。此外透过理论联系实际,巩固所学的知识,提高处理实际问题的潜力,了解设计专题的主要资料,为毕业设计的顺利进行做好充分的准备,并为自
己能顺利与社会环境接轨做准备。

\section{实习岗位和工作状况介绍}
\subsection{实习单位简介}
本人实习单位为美国国家仪器公司中国上海分公司。国家仪器股份有限公司(National Instruments,简称NI或恩艾仪器)是一家美国公司,从事与测试、控
制、设计领域相关的,包括虚拟仪器和电子测试设备等工程软件的开发。著名产品有图形开发环境LabVIEW、C语言虚拟仪器应用系统LabWindows/CVI、集成电
路分析程序NI Multisim等等;硬件产品包括VXI总线、PXI总线、VME总线的框架与模块,IEEE-488接口以及内部整合电路和其他自动化技术的标准。作为一家
自2000年连续11年入选财富杂志百大最佳工作环境名单的公司,国家仪器在全球范围内拥有五千余名员工并在41个国家经营。NI拥有庞大的用户群体——仅在
2004年度,全球就有超过25,000家公司从NI购买了产品。用户的智慧结合NI先进、优质的产品,造就了无数成功的测试测量方案。借助商业化的计算机平台,
用户仅以传统测试测量系统一半乃至十分之一的成本,便可获得与之相同或更出色的功能。现在,NI中国分公司委托国家级计量单位上海计量院为有需求的用
户提供有偿的专业校验与校准服务,颁发校验证书,确保产品的长期测量精度。

%\bibliography{report}
\end{document}